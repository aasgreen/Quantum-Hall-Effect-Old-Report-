\documentclass[12pt]{article}
\author{Adam A. S. Green}
\title{Summer NSERC Research Project 2010}
\usepackage{amssymb, amsmath, float, subfig,caption, hyperref}
\usepackage{epsfig, graphicx}
\usepackage{fullpage}
\usepackage{verbatim}
\usepackage{setspace}
\begin{document}
\bibliographystyle{plain}

\maketitle
\doublespacing
\section{Introduction}
This project intends to make a study of the feasibility of observing the quantum hall
effect in a system where neutral bosons and an effective magnetic field are used.

There are several inherent limitations in this approach, as the magnitude of the effective
magnetic field is currently limited, and has a curvature that would destroy the symmetry
of particles hopping on a lattice. Several approaches have been tried to replicate the effects
of fermions in the bose system, and they will be discussed below. 

\textbf{SETUP}
It will be set up as follows:
Background will contain some background on the Hall Effect, as well as the
Hofstatder Butterfly. It should contain a lot of what I have learned on the
subject (fairly in depth). 

Research:
This section should give a comprehensive overview of all that I have done this
summer.
Should cross reference with notebooks.
\section{Background}

\subsection{The Quantum Hall Effect}
The regular Hall effect should be known to most undergraduates in physics, being a common
expirement done in the undergrad labs.
It is a phenomena that occurs when a current is run through a wire, with a magnetic field 
applied perpendicularly to the direction of the current. Due to the $F = v \times B$ lorentz 
felt by the current, a slight curving of the current will occur. This curving will cause a build
up of charge on one side of the wire, which will cause a voltage to appear across the wires 
width.
%need to review the hall effect-discuss the sign change problem. Give a brief overview.

The Integer Quantum Hall Effect (IQHE) is similar
in nature to the regular Hall Effect.
It is characterized by the plateaus that occur in the Hall Voltage for specific
values of the magnetic field. 

The cause of this is due to the nature of electrons in a magnetic field.
When a magnetic field is present, the electrons will order themselves in Landua levels,
and they will have a specific quantized energy value associated with the radius of the
Landau level it is in.

%More reading on the Quantum Hall Effect (make sure I am right.)

\subsection{The Hofstadter Buttefly}
The Hofstadter Butterfly is a plot of the energy spectrum of a 2D electron gas in a
crystal, as a function of the magnetic flux being put through the crystal.
%need to include a picture of the butterfly.
This ties in with the Hall effect, because as shown by Laughlin~\cite{LAUGHLIN1981} %need to site
, whenever the Fermi energy lies in an energy gap, the Hall conductance is quantized. The Butterfly
has several large bands present- due to its fractal nature, and thus, makes a compelling study 
for the Quantum Hall effect.
Currently, there is some trouble in directly observing the Butterfly in 
experiment. This is because of the tiny size of the lattice in the semiconductors used,
an incredibly large magnetic field would be needed.
However, the same equations that produce the Buttefly (The Harper Equations), are obeyed by a
system of non-interacting bosons in a lattice with a magnetic field.
This has ignited interest in whether a Butterfly could be created with these conditions
~\cite{ZOLLER2003}, and, more to our purpose, whether the Integer Quantum Hall Effect could be
seen in some of the larger gaps present in the structure of the Butterfly.

Recent progress in creating effective magnetic fields for these neutral atoms has revived interest
in this, and it is the purpose of this research further tie in these neutral bose systems with
the quantum hall effect.
% a discussion of the butterfly, how it is physically created, and mathematically solved- 
%different approaches used.

\section{Research}

\subsection{Effective Magnetic Field}
The first line of inquiry was made to see if these 'effective` magnetic fields were sophisticated enough
to produce enough of a band gap to cause the Quantum Hall Effect. The Butterfly only has these 
pronounced gap structures with higher values of the fraction flux plaquette (P/Q). However, recent research
has shown that practical values of P/Q can be reached~\cite{SPIELMAN2009} %Cite the one paper that specifically mentions this
%Sum up literature, highlight the important limitations- what flux/plaquette can we reach?
%Are there ways to counteract the curvature
Unfortunately, this method also produces a non-localized curvature in the potential. This could possibly
disrupt the plane wave Bloch functions that diagonalize the Hamiltonian, and which are key to the structure of the
Butterfly.

However, the method seems promising enough that we can start exploring our idea
in more depth.
\subsection{Solving Harper's Equations}
The second part of my research was understanding enough of quantum field theory to solve Harper's Equations.
This include the following topics:

\begin{itemize}
\item Creation and Anhiliation Operators
\item Commutator Relations
\item The Structure of Fock Space
\item The Aharamov-Bohm Effect
\item Brillioun Zone
\item Wannier States
\item Bloch's Theorem
\end{itemize}
These all played some part in allowing me to understand and solve the Harper's Equations.
The square lattice was the only system that was examined.
\[
\sum^Q_{j=1} \{ [2cos(k_x + 2 \pi j / \phi_0) -E ] \Psi_j +\exp^{-i k_y} \Psi_{j+1} + \exp^{ik_y} \Psi_{j-1} \}
\]

(NB:1, pg 54, 144(this was the better version))
%Briefly discuss the fundamentals in understanding and solving Harper's Equations
%Some field theory, hopping matrix, bloch states

The first case looked at was extremely simple. Essentially, the Harper equations were solved with no
additional assumptions, to show we could replicate Hofstader's original result.
\begin{figure}[H]
	\centerline{\rotatebox{-90}{\scalebox{.65}{\includegraphics{../Pics/butter_1.ps}}}}
	\caption{ The Butterfly generated is a less resolution one, but the fractal structure
	  clearly matches the original created by Hofstader
          The code I used to generate this figure is in:
          Documents/Report/Source/Code/Original/Basic, and it is called
          hofbut.f90.
        }
	\label{Figure 1}
\end{figure}
I originally wrote code in matlab to produce these data files, but later ported it to
Fortran 90.

At this point we didn't check to see if the code could
replicate the band structure expected when there was no magnetic field.
Later on, using a more advanced version of the code, this was rectified.
The resulting graphic was rendered in 3D, but I will try to include it.
The data for the graphic are in:
\\
/home/green/Documents/Report/Source/Code/Exact/SingleParticle/Data

At this point, I also briefly investigated what happened when P = 0, but Q != 1.
The total magnetic field would still equal zero, but the effects of Q would still
be felt in the system.
Sure enough, it produced quite a weird spectrum (which is also in the folder above.).
However, on further discussion, it was decided that this result is pathological.


Once the results were compared to the Hofstatder's and shown to be in agreement,
I went on to investigate the effect that interactions have on the system, using a Mean
Field Theory at first.



\subsection{Interaction Effects}
%This was kind of abandoned- discuss why. 
%talk about how you add interactions to the mean field theory
%why it is tricky, what it means.
Instead of solving the system exactly, we used a Mean Field Theory Approach (MFT).
This approach is based off of the approximation that all of the interactions
that the particles undergo in the system can be averaged out. Then we only
need to consider this average effect on a single particle to obtain
the full dynamics of the system.

This was done using the bose hubbard model, 
\[
-t \sum_{\langle p, q \rangle} b_p^\dagger b_p + \frac{U}{2} \sum_p \hat{n}_p
(\hat{n_p} - 1) +\mu \sum_p n_p
\].
See notes for a complete deviation. \textit{NB:1, pg 87-90}

The resulting formula that was derived, using the Mean Field Theory approximation:
\begin{equation} 
H = -t \sum_{\langle p, q \rangle} b_p^\dagger b_p e^{i \Theta_{p,q}} +
\frac{U}{2} [ (2n_p -1)\hat{n}_p - \bar{n}^2] +\sum_p \mu n_p
\end{equation}

Then the result was taken into k space,
\begin{equation} \label{BHMeq}
H = \int_k  H_0 +\frac{U}{2} (2\bar{n}-1)b_k^\dagger b_k -\frac{U}{2} n N +
\mu N
\end{equation}.

This was the result used to generate Figure \ref{BHMfig}.


\begin{figure}[H]
	\centerline{\rotatebox{-90}{\scalebox{.45}{\includegraphics{../Pics/N100U0_0002.ps}}}}
	\caption{ The number of particles is 100, and the interactions energy is .0002, and as
		seen, all that occurs is the shift. Though dramatic in terms of the energy scale,
		ie. an interactions energy of 0.0002 caused a shift of 100, the internal 
		structure is not altered at all. Similar results are found, with a varying 
		energy shift for different combinations of N, U. This figure was
		produced using code in /home/green/Documents/Report/Source/Code/Original/Interactions}
	\label{BHMfig}
\end{figure}
%FIGURE commented out so it will print

%path to file pjl/NSERC3/Fixed_Q/Interactions

Using the Mean Field Theory approximation to the Bose Hubbard Model shows a
robust gap, even with interactions.

This is not so unexpected though, as seen in~\eqref{BHMeq}, the MFT just gives
an offset on the diagonal terms of the Hamiltonian - resulting in the shift
seen in Figure \ref{BHMfig}.

The result is encouraging, as if the gap remains robust in the presence of
interactions, Laughlin's argument should be applicable.
\subsection{Temperature Effects}
This whole project is predicated on the assumption that we can obtain a system
that is similar to a 2D ferimon gas in a crystal lattice using neutral bosons
in an optical lattice. However, bosons do not behave like fermions.
One boson and one fermion appear indistinguishable, as their identifying
statistics only come to bare when they interact with their fellows. This is
why our boson butterfly was exactly the same as the fermion butterfly.
In reality, we would not be inserting one boson into the system, there would
be quite a few.

In an attempt to mimic the behavior of the repulsive interactions of the fermions
we turned on the temperature. The goal being to try to obtain a configuration
that mimics that of a Fermion at zero temperature. This would allow the
standard methods for calculating that Hall Conductance to be used.~\cite{STREDA1982}~\cite{THOULESS1982}~\cite{2STREDA1982}

First, to get a feeling for how temperature effects the system, I created a
set of animated GIF showing how the energy levels become occupied as
temperature increases. They can be found here:
\\\begin{verbatim}/home/green/Documents/Report/Source/Code/Analysis/Movie/Butterfly_Full_Mu-100\end{verbatim}
Several promising features can be seen in the movies- such as the bottom
energy level filling up nicely, while the second energy level is often empty.

A closer look was taken for a flux of 1/10, to see specifically what was happening.
By observing Figure \ref{tempdist}, we can see that the
bottom band is almost completely filled, while the second band has negligible
occupation.
Almost exactly like a fermionic system at zero temperature with the Fermi
energy in the gap!

In fact, it seems to be temperature which fulfills this role. If we `normalize'
the energy to make it directly comparable to the temperature (cannot have $t <
0)$), then our bottom energy would become 0, and our second energy would
become $-2.35+3.42 = 1.07$;this would put the middle of the gap at around
$.535$. Thus, if we had an system equivalent to a Fermi system, we could
predict that at $T = .535$, bottom band would be completely occupied, while the
top band wouldn't have any occupation at all.

\begin{figure}[H]
	\centerline{\rotatebox{0}{\scalebox{.45}{\includegraphics{../Pics/E1_E2_T2_N10000.eps}}}}
	\caption{The first two energy levels are shown in contrast. More graphs can
	be provided, but the same basic analysis is given.
	Note interesting things are bose clumping found- where the higher the
	number of particles, the more loath they were to leave the first energy level.
	And the split energy levels. The first energy band seems to evolve towards
	having the same occupation probability, making them appear indistinguishable
	in a sense. Would this be the same as having one huge bose condensate? Note the
	second band is still negligibly filled at this point. Flux = 1/10.
	This figure was generated using matplotlib with ipython. I used data files
	generated by the code found here:
        \\ /home/green/Documents/Report/Source/Code/Analysis/BoseStat/Python
        \\Note that at the predicted temperature of T=.535, we have the
        state we are looking for
	}
	\label{tempdist}
\end{figure} %PATHTOFILE: /anon-dhcp12.Documents/Temp_Affects/Less_K/Prob_Graphs/Temp_2_Flux_.1_Num_10000
%path to file pjl/NSERC3/Fixed_Q/Interactions


Promising as this approach is, there are some fairly serious issues with it.
This was done using the grand canonical boson distribution (as the
bose distribution is derived within the grand canonical framework), and there
may be some suspicions in the manner which we use the energy levels.

The energy levels used are non-interacting, single particle.
When using the grand canonical ensemble, we are injecting a possibility of an 
infinite amount of particles into our system. To be sure of our results, we will need
to derive the energy landscape of an interacting, multi-particle system, and
apply the temperature analysis to THAT spectrum.

%Discuss how .gif shows the energy levels being filled as a function of temperature.
%Include the recent figures to show that at low temperatures we have the first band almost 
%completely filled, but the second band is not not filled up at all
%Discuss how N factors into this quite a bit- bose-enhancement (like to clump)
%maybe explore fractured bose condensate that D.F. discussed.
\section{Temperature Effects Exactly}
The next challenge was to try to exactly solve the system for very
small numbers of particles, ie. 2,3,4. 
This required new code to be written, as the system could
no longer be modeled as the energy levels of one particle.

In the absence of interactions, the problem is slightly more
tractable, as it is just the tensor product of the energy
levels of the single particle. The bosons can't see each other,
so they don't perturb the system at all, but the energy you
measure in the system is changed- you need to add up the energy
each boson contributes to the system and account for all the combinations.
\begin{figure}[H]
	\centerline{\scalebox{.45}{\includegraphics{../Pics/energsmear_N1_3.eps}}}
	\caption{ Showing how the energy spectrum changes as more particles
	are allowed into the system. N=1,2,3 were shown to get a sense of the trend.    }
	\label{NEnergyfig}
\end{figure}

The code produced is contained in the directory:
\\/home/green/Documents/Report/Source/Code/Analysis/GrandCanonical/NoInteractions


The code written accurately reproduces the spectrum for N=1, and
as predicted, gave us the same result as the tensor product, see calib.py in
the same folder.
(This was tested with a python script that would take the
tensor product and we compared this spectrum to the one
produced explicitly by the Hamiltonian.

%should show a graph of each, side by side maybe
The next step is to apply temperature, and see if the promising
results predicted by the one particle spectrum hold up.
There is some fudging involved, as the bose distribution requires
a possible infinite amount of particles to be added to the system
(as it is the result of a geometric series resulting from the grand
partition function extended to include an infinite number of particles).

There was also some difficulties in using the grand partition function itself,
as the energy levels could possibly change as a function of how many particles
are in the system (in the interacting case at least).

It was thought that as long as the grand partition function
was truncated at a sufficiently high enough particle number, with respect
to the average particle number selected for by the user, the series would still be accurate enough to predict a gap, or at least give evidence of one.
A python code was written that would calculate values from the grand partition function.
For no interactions, the average energy of the system is simple to realize.
\begin{figure}[H]
	\centerline{\scalebox{.44}{\includegraphics{../Pics/N1_2_3_P1_Q8_Nr_1_7_T1.eps}}}
	\caption{ Without interactions, increasing the N just increases the
	number of possible combinations of energy, so we would expect no
	change to the energy spectrum. Note the nice plataea that qualitatively
	shows a band gap is present.}
	\label{NEnergyavefig}
\end{figure}


%GRAPHS TO SHOW: REPRODUCE N=1
%				 Energy level (showing how the gap seems smeared
%				 Average K value?
%                The Average Energy of the system (derivative appch)
%the ave. energy graph is the method we use to see the first gap.
%note how, even here, the gap is getting smoother.
\section{Gap Test}
The final test is to see if the energy gap seen in the original
butterfly stays robust even after temperature and interactions
are turned on. A system was needed where we could gauge the
size of the band gap. 
\subsection{Toy Model}
The following toy model was proposed:
Two energy levels,$\epsilon_0$, $\epsilon_0 + \Delta$. They were then modeled
using the canonical ensemble.
\begin{equation} \label{toymodel1}
Z = \exp^{- \epsilon_0 / t} + \exp ^{ - (\epsilon_0 + \Delta) / t}
\end{equation}
If the system has a well defined band gap, then the average energy 
should go through some noticeable plateaus. This would be because the
particles would not have enough thermal energy to overcome the gap $\Delta$,
and so, until that threshold temperature was reached, the average energy
should not increase.

However, as we are working in the grand canonical ensemble, another toy 
model was developed in the grand canonical framework.
\textit{NB:2, pg 86-89, 95-99} 
The average energy of our second toy model,
\begin{equation}\label{toyeq}
E_{\mathrm{ave}} = \left(  \left(  \left( {e^{{\frac {\epsilon-\mu}{t}}}}-1 \right) ^{-1
}+ \left( {e^{{\frac {\epsilon+\delta-\mu}{t}}}}-1 \right) ^{-1}
 \right) \epsilon+\delta \left( {e^{{\frac {\epsilon+\delta-\mu}{t}}}}
 -1 \right) ^{-1} \right)  \left(  \left( {e^{{\frac {\epsilon-\mu}{t}}
 }}-1 \right) ^{-1}+ \left( {e^{{\frac {\epsilon+\delta-\mu}{t}}}}-1
  \right) ^{-1} \right) ^{-1}
  \end{equation}
However, assuming that the temperature is low, this can be approximated,
\begin{equation}
E_{mathrm{ave}} \approx \epsilon + \frac{\Delta}{(e^{\Delta/t}+1}
\end{equation}

\begin{figure}[H]
	\centerline{\scalebox{.55}{\includegraphics{../Pics/energyfitplot1.eps}}}
	\caption{ The various approximations are shown for the toy model.
          The grand canonical approximation refers to Eq. \eqref{toyeq}, while
          the canonical approximation refers to Eq. \eqref{toymodel1}
        }
	\label{energycomfig}
\end{figure}

Now, using Eq. \eqref{toyeq}, we can apply it to a very simple case to
see if we can read off the gap, $\Delta$, by trying to fit that model to
the data.
Xmgrace's non-linear curve fitting macro was used.


The way we test for gap robustness is by solving for the average energy
of our truncated grand canonical system as a function of temperature.
If we observe any characteristic plateaus in the data, we can attempt to
fit a curve according to Eq. \eqref{toyeq}. 
If we get a good fit for low temperatures, then we can read the
gap right off of the fitted parameters.
This method will work best when the energy bands are small, allowing
them to be closely approximated by the point energies $\epsilon$ and 
$\epsilon+\Delta$. 
\subsubsection{Testing the Gap Test (No Interactions)}
We have confirmed that for low temperatures and negative epsilons, that the
approximation is good, and fits closely with the toy model. Now, we will
apply it to a case where we know what the gap is, to see if we can recover
$\Delta$.

This allows us to characterize band gaps, even in the presence of interactions,
where the systems energy cannot simply be modeled as the tensor product
of the single particle energy spectrum
\begin{figure}[H]
	\centerline{\scalebox{.35}{\includegraphics{../Pics/energyfittestN1_P1_Q8_t0.6.eps}}}
	\caption{ The toy model was tested against  numerical data created to 
          model the average energy of a grand
          canonical system truncated at 7 particles. It was found upon
          analysis that adding the second parameter, A1, into Eq. \ref{toyeq}
          in place of the 1 in the $(exp(\Delta/t)+1) \rightarrow
          (exp(\Delta/t+A1))$ term increased the approximations accuracy.
          See Table \ref{Nenergytble} for other gap values calculated.
          It was plotted using data created by code found in:
          Documents/Report/Source/Code/Exact/NoInteractions
        }
          
           
	\label{toytestN1}
\end{figure}

\begin{table}[H]
\centering
 \begin{tabular}{c c c} 
   N & Gap & A1 \\ \hline \hline
   1 & 1.28728 & -3.29214 \\
   2 & 1.21323 & -3.15381 \\
   3 & 1.17691 & -2.71397 \\ \hline \hline
   Actual & 1.278 & - \\ \hline
 \end{tabular}
\caption{Even though it looks like there is a trend towards lower precision in
  the gap calculated, there may not be. The N=2, and N=3 values were done using
a $t=0..1$ range, while the N=1 case was done with $t=0..0.6$.
This data was produced by code found in: \\
 Documents/Report/Source/Code/Exact/NoInteractions, and the data was found in
directory: \\
Documents/Exact/TempEffects/GrandCan} \label{Nenergytble}
\end{table}



Table \ref{Nenergytble} is only meant to show that this method gives a
qualitative measure of the robustness of, the gap- not that it can be used to
precisely give a value for the gap.

%plots needed:
%plot showing that approximated works for toy model
%plot showing it gives the right info for known case, N=1
%plot showing the gap robustness is preserved for interactions, N=2 
\subsubsection{Applying the Gap Test (Interactions Turned On)}
We then took our toy model and applied it to numerical data derived with
interactions taken into account.
The code used to generate the figures in this section can be found here:
\\
/home/green/Documents/Report/Source/Code/Exact/Interactions

\bigskip
\begin{figure}[H]
	\centerline{\scalebox{.35}{\includegraphics{../Pics/gaptest_N1_Nr1_7_T0.8_P1_Q8_G0.2.eps}}}
	\caption{ The toy model fits quite well to this data set. The gap
          can be read directly off the graph as $\Delta = 1.01976$, so
          we can suppose that the gap remains robust, at least with one
          particle. This was expected, as with one particle, interactions
          shouldn't really become an issue
        }
        \label{fig:toyapplyN1}
          
\end{figure}

\begin{figure}[H]
	\centerline{\scalebox{.35}{\includegraphics{../Pics/gaptest_N2_Nr1_7_t2.2_2.58_G0.2_P1_Q8.eps}}}
	\caption{ The toy model does not fit well at all, and gives patently
          bad values for the line fit. The BEC temperature calculated
          is too high, and so we lose the valuable low temperature region
          of the graph that is used to calculate the gap.
        }
        \label{fig:toyapplyN2}
\end{figure}


There are several problems with this approach. When the interactions
are turned on, it can make it difficult to access the low temperature
area of the average energy, as in Figure \ref{fig:toyapplyN2}
The reason is discussed below:
\\
The interactions effects grow as a function of N. They shift the energy spectrum
up, as seen in Figure \ref{BHMfig}. This means, that at low temperatures,
the system will prefer to stay in a low particle number configuration- as
this will minimize the energy. If the average number of particles wanted is 
higher than 1, the system must compensate somehow. The chemical potential
cannot increase anymore(as it is fixed at the smallest energy level), so
the only thing left to give is the temperature.
The BEC temperature becomes noticeably quite large when N is increased.
%maybe show table of BEC temp as a function of N
%need to gather more data
As the BEC temperature mandates the lowest temperature we can get to and
still have our particle number conserved, a high BEC temp means we
are pushed out of the desirable low temperature limit that makes our
approximation work so well.
At higher temperatures, the particles can sample more and more of our system,
and this will mean that our simplified toy model with only two energy levels
will no longer be a good fit.
N = 2 seems to be as high up as we can get and still get access to reasonable
temperatures. A promising result, but hardly conclusive.

\section{Kubo}
\textbf{NEED TO FINISH}
All the evidence gathered so far, while not conclusive, seems to suggest
that the large gaps present in the Butterfly remain large, even with
interactions and temperature turned on.
In the methods usually used to calculate the Hall Voltage, the Fermi energy
only comes into play to ensure that all the states below it are completely filled.
As the temperature neatly takes care of this, the formulas developed by
St\u{r}eda and Thouless should be directly applicable to our case.
If there is enough time, an effort will be made to confirm the prediction
of an Integer Hall effect in the system of neutral bosons under investigation.
\section{Conclusion}
We have found several promising hints that the temperature of the
neutral boson system acts equivalently to the Fermi energy of a Fermionic
system.

If true, then following Laughlin's argument, we predict that a Hall
Conductance
would be observed if a potential is applied to the system.
This suggests that if we apply the Kubo formula, we will be able to
predict the Hall conductance in our system.

If we have enough time, we will apply the Kubo argument, a la St\u{r}eda,
and perhaps even test out Thouless's method of calculating the Hall
conductance.


%The promising research into the Temperature Effects merits a more in depth investigation.
%Proposal for looking at solving the hamiltonian exactly
%Problems might be that we want high N to get the really nice systems, and our exact calculations
%will be limited to small N systems.
\bibliography{Master}
\end{document}


